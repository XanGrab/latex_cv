\documentclass[10pt, letterpaper]{article}

% Packages 
% {
\usepackage[
    ignoreheadfoot, % set margins without considering header and footer
    top=2 cm, % seperation between body and page edge from the top
    bottom=2 cm, % seperation between body and page edge from the bottom
    left=2 cm, % seperation between body and page edge from the left
    right=2 cm, % seperation between body and page edge from the right
    footskip=1.0 cm, % seperation between body and footer
    % showframe % for debugging 
]{geometry} % for adjusting page geometry
\usepackage{titlesec} % for customizing section titles
\usepackage{tabularx} % for making tables with fixed width columns
\usepackage{array} % tabularx requires this
\usepackage[dvipsnames]{xcolor} % for coloring text
\definecolor{primaryColor}{RGB}{0, 0, 0} % define primary color
\usepackage{enumitem} % for customizing lists
\usepackage{fontawesome5} % for using icons
\usepackage{amsmath} % for math
\usepackage[
    pdftitle={John Doe's CV},
    pdfauthor={John Doe},
    pdfcreator={LaTeX with RenderCV},
    colorlinks=true,
    urlcolor=primaryColor
]{hyperref} % for links, metadata and bookmarks
\usepackage[pscoord]{eso-pic} % for floating text on the page
\usepackage{calc} % for calculating lengths
\usepackage{bookmark} % for bookmarks
\usepackage{lastpage} % for getting the total number of pages
\usepackage{changepage} % for one column entries (adjustwidth environment)
\usepackage{paracol} % for two and three column entries
\usepackage{ifthen} % for conditional statements
\usepackage{needspace} % for avoiding page brake right after the section title
\usepackage{iftex} % check if engine is pdflatex, xetex or luatex

% Ensure that generate pdf is machine readable/ATS parsable:
\ifPDFTeX
    \input{glyphtounicode}
    \pdfgentounicode=1
    \usepackage[T1]{fontenc}
    \usepackage[utf8]{inputenc}
    \usepackage{lmodern}
\fi

\usepackage{charter}
% }
% Settings:
% { 
\raggedright
\AtBeginEnvironment{adjustwidth}{\partopsep0pt} % remove space before adjustwidth environment
\pagestyle{empty} % no header or footer
\setcounter{secnumdepth}{0} % no section numbering
\setlength{\parindent}{0pt} % no indentation
\setlength{\topskip}{0pt} % no top skip
\setlength{\columnsep}{0.15cm} % set column seperation
\pagenumbering{gobble} % no page numbering

\titleformat{\section}{\needspace{4\baselineskip}\bfseries\large}{}{0pt}{}[\vspace{1pt}\titlerule]

\titlespacing{\section}{
    % left space:
    -1pt
}{
    % top space:
    0.3 cm
}{
    % bottom space:
    0.2 cm
} % section title spacing

\renewcommand\labelitemi{$\vcenter{\hbox{\small$\bullet$}}$} % custom bullet points
\newenvironment{highlights}{
    \begin{itemize}[
        topsep=0.10 cm,
        parsep=0.10 cm,
        partopsep=0pt,
        itemsep=0pt,
        leftmargin=0 cm + 10pt
    ]
}{
    \end{itemize}
} % new environment for highlights

\newenvironment{highlightsforbulletentries}{
    \begin{itemize}[
        topsep=0.10 cm,
        parsep=0.10 cm,
        partopsep=0pt,
        itemsep=0pt,
        leftmargin=10pt
    ]
}{
    \end{itemize}
} % new environment for highlights for bullet entries

\newenvironment{onecolentry}{
    \begin{adjustwidth}{
        0 cm + 0.00001 cm
    }{
        0 cm + 0.00001 cm
    }
}{
    \end{adjustwidth}
} % new environment for one column entries

\newenvironment{twocolentry}[2][]{
    \onecolentry
    \def\secondColumn{#2}
    \setcolumnwidth{\fill, 4.5 cm}
    \begin{paracol}{2}
}{
    \switchcolumn \raggedleft \secondColumn
    \end{paracol}
    \endonecolentry
} % new environment for two column entries

\newenvironment{threecolentry}[3][]{
    \onecolentry
    \def\thirdColumn{#3}
    \setcolumnwidth{, \fill, 4.5 cm}
    \begin{paracol}{3}
    {\raggedright #2} \switchcolumn
}{
    \switchcolumn \raggedleft \thirdColumn
    \end{paracol}
    \endonecolentry
} % new environment for three column entries

\newenvironment{header}{
    \setlength{\topsep}{0pt}\par\kern\topsep\centering\linespread{1.5}
}{
    \par\kern\topsep
} % new environment for the header

\newcommand{\placelastupdatedtext}{% \placetextbox{<horizontal pos>}{<vertical pos>}{<stuff>}
  \AddToShipoutPictureFG*{% Add <stuff> to current page foreground
    \put(
        \LenToUnit{\paperwidth-2 cm-0 cm+0.05cm},
        \LenToUnit{\paperheight-1.0 cm}
    ){\vtop{{\null}\makebox[0pt][c]{
        \small\color{gray}\textit{Last updated in September 2024}\hspace{\widthof{Last updated in September 2024}}
    }}}%
  }%
}%

% save the original href command in a new command:
\let\hrefWithoutArrow\href
% }

\begin{document}
    \newcommand{\AND}{\unskip
        \cleaders\copy\ANDbox\hskip\wd\ANDbox
        \ignorespaces
    }
    \newsavebox\ANDbox
    \sbox\ANDbox{$|$}

    \begin{header}
        \fontsize{14 pt}{14 pt}\selectfont Alexander Grabowski

        \vspace{5 pt}

        \normalsize
        \mbox{Madison, WI}%
        \kern 5.0 pt%
        \AND%
        \kern 5.0 pt%
        \mbox{\hrefWithoutArrow{mailto:ajgrabowski@wisc.edu}{ajgrabowski@wisc.edu}}%
        \kern 5.0 pt%
        \AND%
        \kern 5.0 pt%
        \mbox{\hrefWithoutArrow{https://xangrab.com}{xangrab.com}}%
        \kern 5.0 pt%
        \AND%
        \kern 5.0 pt%
        \mbox{\hrefWithoutArrow{https://www.linkedin.com/in/alexander-xander-grabowski-8164481a1/}{linkedin.com/in/alexander-xander-grabowski-8164481a1}}%
        \kern 5.0 pt%
        \AND%
        \kern 5.0 pt%
        \mbox{\hrefWithoutArrow{https://github.com/XanGrab}{github.com/XanGrab}}%
    \end{header}

    \vspace{5 pt - 0.3 cm}
    
    \section{Education}
        % Entry 1
        %{
        \begin{twocolentry}{
            Sept 2024 – 
        }
        \textbf{M.S. in Curriculum and Instruction}, University of Wisconsin—Madison
        \end{twocolentry}
        \begin{onecolentry}
            \begin{highlights}
                \textit{Focus:} Design, Informal, Creative, Education
            \end{highlights}
        \end{onecolentry}
        % }
        
        \vspace{0.2 cm}
        
        % Entry 2
        %{
        \begin{twocolentry}{
            Sept 2023 – Aug 2024
        }
        \textbf{Capstone Certificate in UX Design}, University of Wisconsin—Madison
        \end{twocolentry}
        \begin{onecolentry}
            \begin{highlights}
                \textit{Focus:} Games User Research \& Design
            \end{highlights}
        \end{onecolentry}
        % }
        
        \vspace{0.2 cm}

        % Entry 3
        %{
        \begin{twocolentry}{
            Sept 2019 – May 2023
        }
        \textbf{Bachelor of Arts in Computer Science}, University of Wisconsin—Madison
        \end{twocolentry}
        \begin{onecolentry}
            \begin{highlights}
                \textit{Certification (Minors):} Software Engineering, Game Development, Digital Art, Japanese Communication
            \end{highlights}
        \end{onecolentry}
        % }

    \section{Development Experience}

        % Entry
        % { 
        \begin{twocolentry}{
            Oct 2024 – 
        }
            \textbf{Design Intern}, Field Day Learning Games -- Madison, WI\end{twocolentry}

        \vspace{0.10 cm}
        \begin{onecolentry}
            \begin{highlights}
                \item Development of home website's front-end and content management features
                \item Implementation of the content management system and quality assurance testing for incoming games in the \textbf{\textit{Vault}} game library
            \end{highlights}
        \end{onecolentry}
        % }

        \vspace{0.2 cm}

        % Entry
        % { 
        \begin{twocolentry}{
            Jan 2022 – May 2023
        }
            \textbf{Software Engineering Intern}, Field Day Learning Games -- Madison, WI
        \end{twocolentry}

        \vspace{0.10 cm}
        \begin{onecolentry}
            \begin{highlights}
                \item Development of core gameplay mechanics and features in games such as \textbf{\textit{Wake}} and unreleased titles
                \item Implementation of continuous integration and delivery systems for games using GitHub Actions 
                \item Quality assurance testing for in-game quests and data validation 
            \end{highlights}
        \end{onecolentry}
        % }

        \vspace{0.2 cm}
    
    \section{Informal Instruction}
        % Entry
        % { 
        \begin{twocolentry}{
            Aug 2024 – 
        }
            \textbf{Student Technology Trainer}, UW—Madison: Division of Information Technology (DoIT)
            \end{twocolentry}

        \vspace{0.10 cm}
        \begin{onecolentry}
            \begin{highlights}
                \item Expanded the instructional content areas to include new workshops and training manuals for game development technologies such as Godot, Unity, and Figma
                \item Developed existing programming materials in Python, JavaScript, and C\# by creating new project-based exercises that increased student engagement and turnout
            \end{highlights}
        \end{onecolentry}
        % }

        \vspace{0.2 cm}
        
        % Entry
        % { 
        \begin{twocolentry}{
            Jun 2024 – Aug 2024
        }
            \textbf{Game Development Instructor}, Maydm Inc -- Madison, WI\end{twocolentry}

        \vspace{0.10 cm}
        \begin{onecolentry}
            \begin{highlights}
                \item Developed introductory programming curriculum for secondary students ages 10+ focused on conditional logic, object-oriented programming, and simple game loops using C\# the Unity game engine
                \item Taught various hard technical skills including visual asset creation, rapid digital and paper prototyping, and multi-platform publishing using game engines.
                \item Introduced students to software development concepts within an iterative design process that emphasized the integration of player feedback and collaboration
            \end{highlights}
        \end{onecolentry}
        % }

        \vspace{0.2 cm}

        % Entry
        % { 
        \begin{twocolentry}{
            Nov 2020 – Jun 2022
        }
            \textbf{Head Programming Instructor}, Code Ninjas -- Sun Prairie, WI
        \end{twocolentry}

        \vspace{0.10 cm}
        \begin{onecolentry}
            \begin{highlights}
                \item Couched youth Fortnite esports team to compete in XPLeague’s North American Finals in 2022.
                \item Developed introductory game development and programming curriculum including materials in JavaScript, Roblox Studio (Lua), and the Unity game engine.
                \item Taught various principles of STEM and IT to a diverse range of kids ages K-12 which included project-based lessons in Scratch and MakeCode Arcade.
            \end{highlights}
        \end{onecolentry}
        % }

    \pagebreak
    \section{Publications}
        
        \begin{samepage}
            \begin{twocolentry}{
                May 20XX
            }
                \textbf{Probably like a Masters thesis title here?}
            \end{twocolentry}

            \vspace{0.10 cm}
            
            \begin{onecolentry}
                \mbox{\textbf{\textit{Alexander Grabowski}}}, \mbox{John Doe}, \mbox{Frodo Baggins}

                \vspace{0.10 cm}
                
        \href{https://doi.org/here}{put.doi.here}
        \end{onecolentry}
        \end{samepage}
    
    \section{Projects}
        % Entry
        % { 
        \begin{twocolentry}
        {
            \href{https://www.vault.fielddaylab.wisc.edu/}{www.vault.fielddaylab.wisc.edu}
        }   
        \textbf{Vault Learning Library} \newline
            \hspace{0.2 cm} \textit{Tools Used:} Squarespace, JavaScript code injection, Google Workspace Suite
        \end{twocolentry}

        \vspace{0.10 cm}
        \begin{onecolentry}
            \begin{highlights}
                \item Guided and managed interns in the process of creating and managing the content in Squarespace 
                \item Implementation of a web form for game studios to publish their educational game on the site
            \end{highlights}
        \end{onecolentry}
        % }

        \vspace{0.2 cm}

        % Entry
        % { 
        \begin{twocolentry}
        {
            \href{https://github.com/fielddaylab/wake}{github.com/fielddaylab/wake}
        }   
        \textbf{Wake: Tales from the Aqualab} \newline
            \hspace{0.2 cm} \textit{Tools Used:} Unity Game Engine, C\#, Firebase, in-house scripting languages
        \end{twocolentry}

        \vspace{0.10 cm}
        \begin{onecolentry}
            \begin{highlights}
                \item Implementation of the modeling features, dialogue, quests, and data analytics of the game
                \item Implementation of user interfaces in Unity 3D and traditional HTML5-Web environments for 2D games
            \end{highlights}
        \end{onecolentry}
        % }

        \vspace{0.2 cm}

        % Entry
        % { 
        \begin{twocolentry}{
            \href{https://github.com/opengamedata/opengamedata-js-log}{github.com/opengamedata}
        }
            \textbf{Open Game Data: JavaScript Client Logging Package} \newline
            \hspace{0.2 cm} \textit{Tools Used:} JavaScript, Node.js, Firebase
        \end{twocolentry}

        \vspace{0.10 cm}
        \begin{onecolentry}
            \begin{highlights}
                \item Developed a Node.js service package for logging data in web games with OpenGameData's servers.
            \end{highlights}
        \end{onecolentry}
        % }

        \vspace{0.2 cm}

    
    \section{Technologies}

        \textbf{Game Engines}
        
        \begin{onecolentry}
            \begin{itemize}
                \item Unity
                \item Godot
                \item Phaser HTML5 framework
            \end{itemize}
        \end{onecolentry}
        
        \vspace{0.2 cm}
        
        \textbf{Vector Illustration Programs}
        
        \begin{onecolentry}
            \begin{itemize}
                \item Figma
                \item Adobe Illustrator 
                \item Inkscape
            \end{itemize}
        \end{onecolentry}
        
        \vspace{0.2 cm}
        
        \textbf{JavaScript Web Libraries}
        
        \begin{onecolentry}
            \begin{itemize}
                \item Node.js runtime ecosystem (npm, webpack, etc)
                \item Graphics rendering libraries (pixi.js, THREE.js, etc)
            \end{itemize}
        \end{onecolentry}
        
        \vspace{0.2 cm}
        
        \textbf{Web Development/Publishing Platforms}
        
        \begin{onecolentry}
            \begin{itemize}
                \item Hugo static web framework and CMS
                \item Wordpress blog and publishing tool
                \item Squarespace website publishing tools
            \end{itemize}
        \end{onecolentry}
        
        \vspace{0.2 cm}
        
\end{document}